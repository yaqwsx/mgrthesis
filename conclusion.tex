\chapter{Conclusion}\label{chap:conclusion}

We presented a brand new platform for metamorphic distributed robots -- the RoFI
platform. The RoFI platform consists of an autonomous modules which can form a
firm mechanical connection between the modules and therefore, create larger,
more sophisticated structures -- RoFI systems.

As the foundational element of the platform, we designed a docking system,
called RoFI dock, which provides a way to autonomously connect two modules, form
a firm mechanical connection, establish communication lines and share power
between the modules. We showed that the RoFI dock can be manufactured by
commonly available means and that it provides reasonably reliable and strong
connection. Also, we designed the dock such that it is a self-contained device
which can be easily integrated into various modules.

On top the docking system, we defined a what is a module of the platform. We
built a formalism to describe the modules and also the whole RoFI systems
consisting of multiple modules. Together with this formalism we also established
a terminology concerning the systems and their reconfiguration, so the
algorithms for reconfiguration can rely on it.

We also proposed and demonstrated a novel approach for communication between the
modules by adapting standard TCP/IP networking with a custom physical layer. The
usage of TCP/IP allows for seamless integration with other networks and possibly
the Internet, leverages robustness of well tested protocols and also allows for
easy adaptation of state of the art network research.

As an application of the overall platform definition we designed and built the
\emph{universal module}, which should serve as a simple building block of RoFI
systems. To support the universal module, we designed \emph{RoFI Driver} -- a
collection of libraries and protocols covering the basic functionality of the
modules. The library includes implementation of dock driver, network interface
or remote firmware update through dock native communication channel. We also
proposed a library to simplify writing of massively aynchronous firmware for
microcontrollers.

\section{Future Work}

The RoFI platform is an ambitious project with goals beyond the scope of this
thesis. In this thesis we took bottom-up approach -- we started with the design
the modules and gradually working up to higher levels of abstraction. In the
near future, we would like to continue in the bottom-up approach and turn the
RoFI dock into self-contained device ready to embed in modules (this boils down
mainly to adding a control circuitry); next we would like to equip the universal
module with control circuitry and accumulators.

Having the hardware setup described above opens the possibility to take the
up-down approach and focus on the algorithmic aspects of the metamorphic modular
robots -- mainly control and reconfiguration. We would like to implement some
already published algorithms and validate them using the hardware setup provided
by the RoFI platform. Then we would like to explore the possibility of
controlling the RoFI systems in a distributed manner as the nature of the
modules seems suitable for distributed control and could leverage the
computational potential of all the robots in the system.

In the long term goals it could be also interesting to explore topics related to
security (how to prevent inter-module communication sniffing and manipulation),
specification (how to encode abstract task like "bring me a ball" into a
formalism suitable for machine processing) or control synthesis (how to find out
the sequence of actions to fulfill given task).

With more applications of the platform, it we would like revisit the design
choices we made and iteratively improve the platform. This might concern e.g.,
the arrangement of the universal module, physical layer for the communication or
the choice of the microcontroller.