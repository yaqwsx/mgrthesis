\chapter{Introduction}\label{chap:introduction}

There are many applications of robotic systems in today's world. The robotic
field has made an enormous progress over the past decade and robots can perform
various challenging tasks. However, vast majority of robotic systems we can find
today is crafted for a given task and lacks versatility. For many applications
being single-task oriented is an advantage due to simplicity and efficiency of
such a solution. However, for tasks like rescue missions, space assemblies or
new planet colonization versatility of robotic systems can be a~crucial feature.

In the past decade researches have been exploring \emph{modular robotic
systems}. These systems are assembled of modules. By reassembling the modules,
new system with different capabilities is formed. This property makes such
system versatile and allows them to fullfil tasks which cannot be tested in
advance (planet colonization) or feature unpredictable events (rescue missions).

Research in this area runs in two parallel branches -- on one hand there are
publications dealing with hardware design of the modules, on the other hand
there are publications dealing with the algorithms for control synthesis and
reconfiguration of the systems. However, interconnection of these two research
branches occurs only spuriously. We guess this is due two factors;
\begin{enumerate*}
    \item the hardware publications often lack proper documentation as it is
    usually confidential and there is no possibility to get physical modules for
    verification of algorithms.
    \item the algorithmic publication often omit details of the physical
    limitation of the modules.
\end{enumerate*}

Therefore, in this thesis we aim for an ambitious goal -- to introduce a new
modular robotic platform, the \emph{RoFI} platform, specially designed to
overcome the gap between algorithms and the physical world\footnote{We are aware
of the potential introduction of a new dead platform as nicely illustrated at
\url{https://xkcd.com/927/}.}.

The thesis is structured as follows; rest of this chapter gives an overview of
the modular robotic systems. In chapter \ref{chap:rofi} we define the RoFI
platform and in chapter \ref{chap:universal_module} we introduce the basic
module in the platform -- \emph{the universal module}. These chapters present
the outcome of our work. The chapter \ref{chap:behind} shows briefly the
evolution of the platform and the chapter presents the current progress in
prototype implementation \ref{chap:prototypes} of the universal module.

\section{Modular Robots}

A modular robotic platform is a way to build robots consisting of
\emph{modules}, as the name suggests. For our purposes, we consider a module to
be a single unit following a specification given by the platform. Modules are
rather high-level pieces with a certain level of self control instead of
low-level components like individual actuators or sensors. It might even make
sense to talk about modules as individual robots, which are used to build other
robots\cite{brunete_current_2017}.

Each of these modules has a given set of capabilities. By joining multiple
modules and via their cooperation, new capabilities can emerge. Different
configurations of modules can emerge different capabilities. The modules are
usually mechanically connected together to form a single robot.

The mechanical connection of the modules can be done externally, e.g. by an
operator, or can be performed by the modules on each own. In the later case, we
talk about \emph{self-reconfigurable} modules \cite{brunete_current_2017}.
Depending on the topology of the connection, there is a naming established in
the literature\cite{brunete_current_2017}:
\begin{enumerate*}
    \item \emph{chain type} for a linear, snake-like and tree-like
    configurations,
    \item \emph{lattice type} for a regular grid-based robots,
    \item \emph{hybrid type} for robots combining both previous approaches.
\end{enumerate*}
Further, if there is only a single or a few types of modules in the system, the
system is called \emph{metamorphic}\cite{brunete_current_2017}. Modules of such
a system are also called \emph{cells} as they mimic cells in living organisms.

The system can be either \emph{centrally controlled} by a single (and possibly
external) unit, or the distributed nature of the modules can be leveraged and
therefore, the robot can feature \emph{distributed control}. The centrally
controlled approach is considered as an easier one, however it does not utilize
all the potential computational power of the modules and is harder to make
fault-tolerant in case of failure of the control unit compared to the
distributed control.

To help to build an intuition about the modular robots we give an analogy with
Replicators -- robots present in a sci-fi TV series Stargate
SG-1\cite{wright_stargate_1997}. Readers unfamiliar with the TV series can skip
the following paragraph.

Replicators consist of a single Replicator block, which is unable to perform any
action on its own. Single replicator block maps to a module in terminology given
above. However, when multiple blocks are combined, they are able to perform
movement and self-control. Therefore, Replicators are:
\begin{enumerate*}
    \item modular (they can be assembled in many configurations from a given
    building blocks)
    \item self-reconfigurable (the configuration can be changed by the blocks on
    their own) and
    \item metamorphic (as there is only a single type of block).
\end{enumerate*}
Whether Replicators are distributed is unclear -- the series does not give much
detail about it. We strongly believe so, as each blob of modules can operate
independently on the others an in case of reconfiguration all newly emerged
blobs become independent.

\section{Existing Metamorphic Robots}

We provide a selection of few existing projects showing the areas of interest in
current research:

\paragraph{M-TRAN} is a project first published in year 2002
\cite{murata_m-tran:_2002}, followed by the second version in year 2003
\cite{haruhisa_kurokawa_m-tran_2003} and the third version in year 2008
\cite{kurokawa_distributed_2008}. Since that the project appears dead. M-TRAN
module shape is clearly inspiration for other projects (including RoFI). The
project lacks any sources that can be used to by a third party. There is an
attempt of rebuilding the project in an open-source way under name
\emph{Dtto}\cite{noauthor_dtto_nodate}, however, it does not reach the qualities
of the original M-TRAN.

\paragraph{Roombots} \cite{bonardi_locomotion_2012} is a project aiming for
smart furniture. Roombots feature original arrangement of theirs degrees of
freedom. It is also worth noting that the project heavily uses passive
components in its assemblies.

\paragraph{SMORES} \cite{davey_emulating_2012} differs from the other project by
focusing not only on locomotion of the whole system, but also on locomotion of
individual modules. This project

\paragraph{HyMod} \cite{gros_hymod:_2018} is a platform similar to SMORES but it
focuses on solving several technical imperfection like docking system strength
or lack of power sharing between the module in the SMORES.
