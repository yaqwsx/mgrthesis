\chapter{RoFI driver interface}\label{chap:rofi_interface}

The interface of the RoFI driver can be obtained by calling
\cpp{auto interface = rofi::getInterface()}. The
function returns an implementation defined object fulfilling concept
\cpp{RoFI}. The \cpp{RoFI} concept is an object such that:
\begin{itemize}
    \item \cpp{interface.joints} is a container-like object which
    \cpp{value_type} fulfills concept \cpp{Joint}.
    \item \cpp{interface.docks} is a container-like object which
    \cpp{value_type} fulfills concept \cpp{Dock}.
    \item \cpp{interface.self} is an object fulfilling concept \cpp{Module}.
\end{itemize}

\noindent Object \cpp{joint} fulfills concept \cpp{Joint} if:
\begin{itemize}
    \item \cpp{joint} can be copied (\cpp{Joint} is only a facade).
    \item \cpp{float Joint::max() const} returns maximal join position in
    radians.
    \item \cpp{float Joint::min() const} returns minimal joint position
    in radians.
    \item \cpp{float Joint::maxSpeed() const} returns maximal joint speed
    in radians per second.
    \item \cpp{float Joint::maxTorque() const} returns the
    torque capability of joint in \si{\newton\meter}.
    \item \cpp{float Joint::getPosition() const} returns joint position.
    \item \cpp{float Joint::getSpeed() const}
    \item \cpp{void Joint::setTorque( float torque )} sets joint into
    torque movement control mode.
    \item \cpp{void Joint::setSpeed( float speed )} sets joint into
    speed control mode (constant velocity is kept) with given speed.
    \item \cpp{Pub< float > Joint::setPosition( float position ) }
    sets joint into position control mode (position is kept). After the position
    is reached, the position emitted and stream is closed.
    \item \cpp{Pub< float > Joint::onPosition( float position ) const} emits
    message whenever joint is in given position (tolerance is implementation
    defined). The stream never ends.
    \item \cpp{SubPub< T, float > Joint::positionReached( float position ) const }
    after receiving a value of type \cpp{T} emits a \cpp{position} when the
    position is reached.
    \item concurrent commands are sequentialized.
    \item if one command aborts the previous one, error is emitted.
\end{itemize}

\noindent Object \cpp{dock} fulfills concept \cpp{Dock} if:
\begin{itemize}
    \item \cpp{dock} can be copied (\cpp{Dock} is only a facade).
    \item \cpp{Pub< State > Dock::state() const } returns dock current state.
    \item \cpp{Pub< State > Dock::connect() } expands the dock. A value is
    emitted one the final position is reached.
    \item \cpp{Pub< Void > Dock::waitForConnection()} waits in expanded position
    for the mating side. Emits value when mating side connects, then the stream
    closes.
    \item \cpp{Pub< Void > Dock::waitForDisConnection()} Emits value when mating
    side disconnects, then the stream closes. If there is no mating side, emits
    immediately.
    \item \cpp{Pub< Void > Dock::onConnection()} Emits value when mating side
    connects. Never closes.
    \item \cpp{Pub< Void > Dock::onDisconnection()} Emits value when mating side
    connects. Never closes.
    \item \cpp{Pub< MutualOrientation > Dock::getMutualOrientation()} returns
    mutual orientation of docks.
    \item \cpp{Pub< Void > Dockd::disconnect()} retracts the dock. Emits once
    target position is reached.
    \item \cpp{Pub< Void > Dock::connect( Line line ) } connect the EXT or the
    INT line. Once connected, value is emitted.
    \item \cpp{Pub< Void > Dock::disconnect( Line line ) } connect the EXT or
    the INT line. Once connected, value is emitted.
    \item \cpp{Pub< float > Dock::current( Line line ) } returns the current in
    Ampers at given line.
    \item \cpp{Pub< float > Dock::voltage( Line line ) } returns the voltage in
    Volts at given line.
\end{itemize}
Note that the high level interface does not provide a way to send data to
neighbors. The user application should rely on TCP/IP instead.

Object \cpp{module} fulfills concept \cpp{Module} if:
\begin{itemize}
    \item \cpp{module} can be copied (\cpp{Module} is only a facade).
    \item \cpp{Guid Module::getId() const} returns module GUID.
    \item \cpp{std::string Module::getType() const} returns a human-readable
    module type name.
    \item \cpp{ShapeDescriptor Module::getShape() const } returns shape
    descriptor.
\end{itemize}