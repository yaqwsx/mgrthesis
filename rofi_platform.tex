\chapter{The RoFI platform}\label{chap:rofi}

The goal of the RoFI platform is to create a new platform for distributed,
modular and self-reconfigurable robots. The platform does not target real-world
usage but it should serve as a tool for validation of various control algorithms
in a physical world. This chapter presents the platform and gives an in-depth
specification of it.

First, we give a brief introduction to a concept of modular robots and establish
several terms. Then we specify the goals of the RoFI platform and provide its
specification. The specification is given as it is without any reasoning behind
the design choices. The reasoning can be found in the following chapter
\ref{chap:behind}.

\section{Modular Robots}

An modular robotic platform is a way to build robots consisting of
\emph{modules}, as the name suggests. For our purposes, we consider a module to
be a single unit following a specification given by the platform. Modules are
rather high-level pieces with a certain level of self control instead of
low-level components like individual actuators or sensors. It might even make
sense to talk about modules as individual robots, which are used to build other
robots\cite{brunete_current_2017}.

Each of these modules has a given set of capabilities. By joining multiple
modules and via their cooperation, new capabilities can emerge. Different
configurations of modules can emerge different capabilities. The modules are
usually mechanically connected together to form a single robot.

The mechanical connection of the modules can be done externally, e.g. by an
operator, or can be performed by the modules on each own. In the later case, we
talk about \emph{self-reconfigurable} modules \cite{brunete_current_2017}.
Depending on the topology of the connection, there is a naming established in
the literature\cite{brunete_current_2017}:
\begin{enumerate*}
    \item \emph{chain type} for a linear, snake-like and tree-like configurations
    \item \emph{lattice type} for an regular grid-based robots
    \item \emph{hybrid type} for robots combining both previous approaches
\end{enumerate*}
Further, if there is only a single or a few types of modules in the system, it
is called \emph{metamorphic}\cite{brunete_current_2017}. Modules of such system
are also called \emph{cells} as they mimic cells in living organisms.

As the robot is form, it can be either \emph{centrally controlled} by a single
(and possibly external) unit or the distributed nature of the modules can be
leveraged and therefore, the robot can feature \emph{distributed control}. The
centrally controlled approach is considered as an easier one, however it does
not utilize all the potential computational power of the modules and is harder
to make fault-tolerant in case of failure of the control unit compared to the
distributed control.

To help to build an intuition about the platform we give an analogy with
Replicators -- robots present in a sci-fi TV series Star Gate. Readers
unfamiliar with the TV series can skip the following paragraph.

Replicators consist of a single Replicator block, which is unable to perform any
action on its own. Single replicator block maps to a module in terminology given
above. However, when multiple blocks are combined, they are able to perform
movement and self-control. Therefore, Replicators are:
\begin{enumerate*}
    \item modular (they can be assembled in many configurations from a single
    type of unit)
    \item self-reconfigurable (the configuration can be changed by the blocks on
    their own) and
    \item metamorphic (as there is only a single type of block).
\end{enumerate*}
Whether Replicators are distributed is unclear -- the series does not give much
detail about it. We strongly believe so, as each blob of modules can operate
independently on the others an in case of reconfiguration all newly emerged
blobs become independent.

\section{Goals of the RoFI Platform}

The goal of the RoFI platform is to give an reasonably easy way to validate
various control algorithms for modular self-reconfigurable robots, as we
mentioned in the introduction. It does not target for any specific real-world
usage, like e.g. the Roombots\cite{bonardi_locomotion_2012} for being a smart
furniture. To fullfil these goals we give a following list of requirements for
the platform:

\begin{itemize}
    \item All sources (including CAD models, PCBs, firmwares and libraries)
    should be kept open-source, to allow for reproducibility of experiments
    performed on the platform.
    \item It should be manufacturable by a commonly accessible machinery.
    \item No deep understanding of mechanical nor electrical engineering should
    be required to use it.
    \item Clear specification of the modules should be available to allow for
    further extension.
    \item The platform should provide a way to easily distribute new firmware to
    the modules.
    \item The platform should allow for both, central and distributed control.
    \item The platform should provide an easy way to port control algorithms.
    This reduces to a requirement for an easy way to define an atomic control
    actions used by the algorithms.
\end{itemize}

\section{The RoFI Platform Specification}

The RoFI platform is a lattice type modular, self-reconfigurable and metamorphic
platform. It is based on a 10cm cube grid. The platform defines \todo{More and more}

\subsection{The RoFI Module}

\subsection{Lock System}

\subsection{Intermodule Communication and Power Sharing}

\subsection{Intramodule Architecture}

\subsection{The RoFI "BIOS" \todo{Proper naming}}

\subsection{RoFI lib}

\todo{Something here}
